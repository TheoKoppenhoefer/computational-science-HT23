
% Indicate to rubber that there are external files
% rubber: shell_escape


\input{../Latex_Templates/Preamble_Report}

%%%%% TITLE PAGE

\usepackage[margin=1.3in]{geometry}

%\subject{, VT23}
\title{ Report for the Course Modelling in Computational Science, HT23 \\[1ex]
	  \large Project 3: Biome classification}
%\subtitle{}
\author{Theo Koppenhöfer \\[1ex] (with Anna and Carmen, Group 4)}
\date{Lund \\[1ex] \today}

\addbibresource{bibliography.bib}

\graphicspath{{../Project3RandomForestML/plots/}}

\pgfplotsset{
	compat=newest,
    every axis/.append style={
        axis y line=left,
        axis x line=bottom,
        scale only axis,
        % line width=2pt,
%    	max space between ticks=25pt,
        width=0.7\textwidth,
        scaled ticks=true,
        axis line style={thick,-,>=latex, shorten >=-.4cm},
    		x tick label style={
		    /pgf/number format/precision=3
		    }
    },
    every axis plot/.append style={very thick},
    tick style={black, thick},    
}


%%%%% The content starts here %%%%%%%%%%%%%

\usepackage{pythonhighlight}

\begin{document}

\maketitle

\section{Introduction}
The following report is part of the third project of the course Modelling in Computational Science, BERN01, taken at Lund university.
In this project we will use machine learning to classify biomes based on climate and soil data. We will test the performance of our machine
learning model in binary and multiclass classification. We will also compare our model with LPG GUESS output
and modify our model to predict continuous outputs of LPG GUESS.
For this we will discuss the choice of regions and biomes, the setup of our model, give some interesting results, discuss these and finally give a conclusion.
The code to the project was implemented in a \pyth{jupyter notebook}.
The project report and code can be found online under~\cite{Repository}.


% `net primary productivity' (\emph{NPP}) and `vegetation carbon pool' (\emph{VegC})

\section{Methods}

For the binary classification model we chose the biomes `arid shrub' and `desert'. For the choice of regions for training and testing
we  choose two countries which contained sufficient samples of both biomes.
In figure \ref{pl:shrub_desert_country_list} we plot the amount of shrub and desert data points in countries with at least
30 samples of each biome.
\begin{wrapfigure}{r}{0.5\textwidth}
  \centering
  \includegraphics[width=0.5\textwidth]{shrub_desert_country_list.pdf}
  \caption{Amount of shrub and desert landscape in selected countries.}
  \label{pl:shrub_desert_country_list}
\end{wrapfigure}
Our initial choice of regions was Egypt (EGY) and China (CHN).
It turned out however that when we took out the LPG GUESS output from the training data, our model could not handle the classification well.
The reason is that deserts in these countries have very different climates. Thus we chose Egypt for training and Libya (LBY) for testing.
The variable we trained for was the observed biome \pyth{Biome_obs} which was determined with the help of satellite data.

For the multiclass classification we initially chose Africa and China but that too turned out to be a poor choice.
Thus we switched to the regions to Russia for training and Canada for testing. We trained and tested for both
\pyth{Biome_obs} and the LPG GUESS biome classification \pyth{Biome_cmax}.

For the regression model we chose Canada to train and Russia to test the model. The reason for this switch of roles lies in the performance
of the training. The LPG GUESS outputs we trained the model for were \pyth{NPP} (net primary productivity) and \pyth{VegC} (vegetation carbon pool).

If not otherwise stated we used as training data all climate data in the file \pyth{'data_index_2.csv'} together the soil data \pyth{clay}, \pyth{silt}, \pyth{sand} and \pyth{orgC}.
The climate data includes the standard deviation, mean and median over the four seasons for radiation,
precipitation and minimum, maximum and mean temperature measurements over the years 1961-1990.

For the implementation we made extensive use of the \pyth{sklearn} library. We implemented the classification model with \pyth{RandomForestClassifier} and
analysed the permutation importance with \pyth{permutation-_importance}.
The hyperparameter tuning was initially implemented with \pyth{GridSearchCV} and after some bad initial results we switched to \pyth{HalvingGridSearchCV}
(the results did not improve).
We varied the parameters \pyth{max_depth}, \pyth{n_estimators}
\pyth{min_samples_leaf} each between 4 reasonably chosen values around their default.
The regression model was implemented using \pyth{RandomForestRegressor}.
The clustering was performed according to \cite{Clustering}.
To analyse the importance of features we also implemented a crude routine which runs the model whilst dropping some features
and then plots the results for each run.


\section{Results}

In this section we will first discuss our results for the binary classification, then for the multiclass classification
and finally the regression problem.

\subsection{Binary classification}

In figure \ref{pl:egypt_libya_map} one can see the geographical distribution of the biomes in Egypt and
 Libya. As expected the desert is inland whereas the dry shrub is closer to the sea.
\begin{figure}[h]
  \centering
  \begin{minipage}{0.5\textwidth}
    \centering
    \includegraphics[width=\textwidth]{egypt_libya_map.pdf}
    \caption{Biome map for Egypt and Libya.}
    \label{pl:egypt_libya_map}
  \end{minipage}
  \hfill
  \begin{minipage}{0.4\textwidth}
    \centering
    \includegraphics[width=\textwidth]{s3_basic_error_map.pdf}
    \caption{Map of biomes which our model classified wrongly.}
    \label{pl:s3_basic_error_map}
  \end{minipage}
\end{figure}

\begin{figure}[h]
  \centering
  \begin{minipage}{0.45\textwidth}
    \centering
    \includegraphics[width=\textwidth]{climate_egypt_aridShrub.pdf}
    \caption{Mean temperature, precipitation and radiation in Egyptian shrubs.}
    \label{pl:climate:egypt:aridShrub}
  \end{minipage}
  \hfill
  \begin{minipage}{0.45\textwidth}
    \centering
    \includegraphics[width=\textwidth]{climate_egypt_desert.pdf}
    \caption{Mean temperature, precipitation and radiation in Egyptian deserts.}
    \label{pl:climate:egypt:desert}
  \end{minipage}
\end{figure}
In figures \ref{pl:climate:egypt:aridShrub} and \ref{pl:climate:egypt:desert}
we see the time evolution of the mean temperature, precipitation and radiation
in the Egyptian shrub and desert landscapes over a year.
In both climates the temperature, precipitation and radiation peak around the summer.
At closer inspection one can spot subtle differences between the diagrams for the different
variables. We note however that the mean radiation levels look almost identical.
Although not shown here the climate plots for Libya are quite similar.

We also plotted the distribution of the various soil features in figure \ref{pl:egypt_shrub_desert_soil}.
One can see that the soil features do not differ much between the Egyptian desert and shrub.
\begin{figure}[h]
  \centering
  \includegraphics[width=0.6\textwidth]{egypt_shrub_desert_soil.pdf}
  \caption{Distribution of the various soil features for the Egyptian desert and shrub.}
  \label{pl:egypt_shrub_desert_soil}
\end{figure}

When we tested our trained model on Libya we got the results depicted in the confusion table 
\ref{tb:s3_basic_confTable}. The accuracy of our model was approximately 0.96 which is reasonably good. The balanced accuracy is with
approximately 0.85 slightly more modest.
The reason for this lies in the bad recall rate for the arid shrub biome.
In the map \ref{pl:s3_basic_error_map} one can see that all misclassified regions lie at the boundary of
the desert and shrub biomes.
\begin{wraptable}{r}{0.37\textwidth}
  \centering
  \begin{tabular}{rrrrr}
    \toprule
    Truth & arid shrub & desert \\
    Predicted & & \\
    \midrule
    arid shrub & 46 & 0 \\
    desert & 20 & 443 \\
    \bottomrule
  \end{tabular}
  \caption{Confusion table.}
  \label{tb:s3_basic_confTable}
\end{wraptable}

Now we analyse the importance of features for the binary classification.
In figure \ref{pl:histogramm_feature_imps3_basic} 
the MDI (random forest feature importance) shows that
the precipitation levels played an important role in training and the soil played an insignificant role.
Note also that the decrease in accuracy score for all features in the permutation importance is insignificant.
\begin{figure}[h]
  \centering
  \includegraphics[width=0.8\textwidth]{histogramm_feature_imps3_basic.pdf}
  \caption{The five most and least important features.}
  \label{pl:histogramm_feature_imps3_basic}
\end{figure}
\begin{figure}[h]
  \centering
  \begin{minipage}{0.45\textwidth}
    \centering
    \includegraphics[width=0.8\textwidth]{histogramm_feature_imp_clustereds3_basic.pdf}
    \caption{Feature importance after clustering.}
    \label{pl:histogramm_feature_imp_clustereds3_basic}
  \end{minipage}
  \hfill
  \begin{minipage}{0.45\textwidth}
    \centering
    \includegraphics[width=0.9\textwidth]{s3_simulationComparisons_balancedErrorRate.pdf}
    \caption{Results for experiment series.}
    \label{pl:s3_simulationComparisons_balancedErrorRate}
  \end{minipage}
\end{figure}

We also created a dendogram which for space reasons has to be admired in the appendix.
Unsurprisingly we found that the medians and means were strongly correlated. One could also see that the 
soil data was relatively independent from the rest of the data.
Notably the standard deviation of the temperatures in the summer and fall were also quite independent.
The results from the following clustering are then shown in figure \ref{pl:histogramm_feature_imp_clustereds3_basic}.
Here one can notice that the soil features are relatively unimportant as is the cluster including the mean radiation in the summer.

The results of our routine dropping various features can
be seen in figure \ref{pl:s3_simulationComparisons_balancedErrorRate}.
On the y-axis we depict the balanced error rate rather than the classical error rate but the qualitative behaviour is identical.
The abbreviations `pre', `tmp$|$tmin$|$tmax' and `tswrf' correspond to data representing
precipitation, temperatures and radiation respectively.
Firstly one sees that most modifications have little impact.
When we drop all the climate data and only train our model on the soil data it performs very badly.
Surprisingly, dropping the spring, temperature or radiation data
significantly improves the performance of our model.


\subsection{Multiclass classification}

We start by giving an overview of the biomes which were classified. Figure \ref{pl:Russia_Canada_map}
shows the geographical distribution of biomes in Russia and Canada. One sees that the order in which the
biomes appear in Canada as one travels northwards is quite similar to the order in which the biomes appear in
Russia as one travels northeast. Although not shown here the climate diagrams for Russia and Canada are quite similar.
\begin{figure}[h]
  \centering
  \includegraphics[width=\textwidth]{Russia_Canada_map.pdf}
  \caption{Overview of the biomes \pyth{Biome_obs} in Russia and Canada.}
  \label{pl:Russia_Canada_map}
\end{figure}

\begin{wraptable}{r}{0.5\textwidth}
  \vspace*{-0.3cm}
  \centering
  \begin{tabular}{rrrrr}
    \toprule
     & precision & recall \\
    \midrule
    macro average & 0.57 & 0.47 \\
    weighted average & 0.85 & 0.85 \\
    \bottomrule
  \end{tabular}
  \caption{Average precision and recall for the multiclass classification.}
  \label{tb:s4_precision_recall}
  \vspace*{-1cm}
\end{wraptable}
The results for the multiclass classification are shown in table \ref{tb:s4_precision_recall}.
The accuracy of 0.85 is worse than for the binary classification but not by much.
The balanced accuracy of about 0.47 is on the other hand significantly worse.
Once again a map of the misclassified regions can be seen in \ref{pl:s4_basic_error_map} and
we note that these regions tend to lie between two biomes.
\begin{figure}[h]
  \centering
  \begin{minipage}{0.47\textwidth}
    \centering
    \includegraphics[width=\textwidth]{s4_basic_error_map.pdf}
    \caption{Misclassified regions for \pyth{Biome_obs}.}
    \label{pl:s4_basic_error_map}
  \end{minipage}
  \hfill
  \begin{minipage}{0.47\textwidth}
    \centering
    \includegraphics[width=\textwidth]{Basic_with_Biome_Cmax_error_map.pdf}
    \caption{Misclassified regions for \pyth{Biome_Cmax}.}
    \label{pl:Basic_with_Biome_Cmax_error_map}
  \end{minipage}
\end{figure}

The results regarding MDI and permutation importance are shown in figure \ref{pl:histogramm_feature_imps4_basic}.
The dendogram once again was banished to the appendix.
The results of our own set of experiments dropping selected features are shown in figures \ref{pl:s4_simulationComparisons_balancedErrorRate}.
Here dropping the climate and the temperature has the biggest negative impact.
Dropping the winter and the soil on the other hand have a mild positive impact.
\begin{figure}[h]
  \centering
  \includegraphics[width=0.8\textwidth]{histogramm_feature_imps4_basic.pdf}
  \caption{The five most and least important features.}
  \label{pl:histogramm_feature_imps4_basic}
\end{figure}
\begin{figure}[h]
  \centering
  \begin{minipage}{0.45\textwidth}
    \centering
    \includegraphics[width=0.8\textwidth]{histogramm_feature_imp_clustereds4_basic.pdf}
    \caption{Feature importance after clustering.}
    \label{pl:histogramm_feature_imp_clustereds4_basic}
  \end{minipage}
  \hfill
  \begin{minipage}{0.45\textwidth}
    \centering
    \includegraphics[width=0.9\textwidth]{s4_simulationComparisons_balancedErrorRate.pdf}
    \caption{Results for experiment series.}
    \label{pl:s4_simulationComparisons_balancedErrorRate}
  \end{minipage}
\end{figure}

Regarding hyper parameter tuning our results were quite mixed.
When we initially used Africa and China as training and test regions the hyperparameter tuning could
in certain circumstances improve our model. In most cases it tended to overfit if the ranges for the hyperparameters
was not chosen carefully. The overfitting was visible in a very high accuracy on the train and a very low accuracy
on the test set.
For Russia and Canada the hyperparameter tuning did not improve the performance our model. 
Here too it tended to overfit.

We also trained and ran the model to predict the LPG guess output \pyth{Biome_Cmax}. Due to the strict page limit
the author will have to spare the reader the rather uninteresting details but we note that the accuracy and the 
weighted accuracy were with 0.72 and 0.57 high. We also give in figure \ref{pl:Basic_with_Biome_Cmax_error_map}
a map showing the results. One should note here the large difference between the biome types
to \pyth{Biome_Cmax} in figure \ref{pl:s4_basic_error_map}. 
More interestingly we tested a model trained on \pyth{Biome_obs} with \pyth{Biome_Cmax} data. Here the accuracy
plummeted to 0.14 and the balanced accuracy to 0.31.

When we tested a model trained on \pyth{Biome_Cmax} on \pyth{Biome_obs} the results were similarly abysmal and will be skipped here.

\subsection{Regression}

We begin by plotting the distribution of the parameters \pyth{NPP} and \pyth{VegC} in both Canada and Russia. The results can be
seen in figure \ref{pl:npp_vegc_distribution}. We see that \pyth{VegC} has a lot of very small values whereas \pyth{NPP} is more spread out
in both domains. We also note that the values for \pyth{VegC} are about an order of magnitude larger than those of NPP.
\begin{figure}[h]
  \centering
  \includegraphics[width=0.7\textwidth]{npp_vegc_distribution.pdf}
  \caption{Distribution of \pyth{NPP} and \pyth{VegC} in Canada and Russia.}
  \label{pl:npp_vegc_distribution}
\end{figure}

We first trained the model for \pyth{NPP} and tested it on Russia. The results 
can be seen in the scatterplot \ref{pl:s5:npp:basic:regressionPlot}. It can also be seen in the distribution
 of the residues $\hat{Y}_\text{test}-Y_\text{test}$ in figure \ref{pl:s5:npp:basic:residualDistr}.
\begin{figure}[h]
  \centering
  \begin{minipage}{0.45\textwidth}
    \centering
    \includegraphics[width=\textwidth]{s5_npp_basic_regressionPlot.pdf}
    \caption{Predicted versus true values for the parameter NPP.}
    \label{pl:s5:npp:basic:regressionPlot}
  \end{minipage}
  \hfill
  \begin{minipage}{0.45\textwidth}
    \centering
    \includegraphics[width=\textwidth]{s5_npp_basic_residualDistr.pdf}
    \caption{Distribution of the residues for the parameter NPP.}
    \label{pl:s5:npp:basic:residualDistr}
  \end{minipage}
\end{figure}
We also plotted a map of the residues seen in figure \ref{pl:s5_npp_basic_error_map}. From this one sees that the model works best in the north and east of Russia.
\begin{figure}[h]
  \centering
  \includegraphics[width=0.7\textwidth]{s5_npp_basic_error_map.pdf}
  \caption{Map of the residue.}
  \label{pl:s5_npp_basic_error_map}
\end{figure}

After this we trained and tested the model for VegC.
The plots of the residuals and the predicted versus true values are very similar to \pyth{NPP} which is why
they will will not be discussed here and the keen reader is referred to the appendix.

We will proceed in taking a closer look at the model predicting NPP.
The feature importance can be seen in figures \ref{pl:histogramm_feature_imps5_npp_basic} and after clustering in
\ref{pl:histogramm_feature_imp_clustereds5_npp_basic}.
Note that the temperature in the spring, summer and fall the most important factors for
for the training data. For the test radiation is the most important factor.
\begin{figure}[h]
  \centering
  \includegraphics[width=0.8\textwidth]{histogramm_feature_imps5_npp_basic.pdf}
  \caption{The five most and least important features.}
  \label{pl:histogramm_feature_imps5_npp_basic}
\end{figure}
\begin{figure}[h]
  \centering
  \begin{minipage}{0.45\textwidth}
    \centering
    \includegraphics[width=0.8\textwidth]{histogramm_feature_imp_clustereds5_npp_basic.pdf}
    \caption{Feature importance after clustering.}
    \label{pl:histogramm_feature_imp_clustereds5_npp_basic}
  \end{minipage}
  \hfill
  \begin{minipage}{0.45\textwidth}
    \centering
    \includegraphics[width=0.9\textwidth]{s5_simulationComparisons_MSE.pdf}
    \caption{Results for experiment series.}
    \label{pl:s5_simulationComparisons_MSE}
  \end{minipage}
\end{figure}
For a dendogram of this experiment we refer to the appendix.

The results of our own set of experiments is shown in figure \ref{pl:s5_simulationComparisons_MSE}. Here we show the mean square error
on the y-axis. We created the same chart also for the $R^2$ error, the mean absolute error and the maximal error but the qualitative behaviour
for these different error metrics was the same.
Once again we remark on the outliers.
Unsurprisingly the experiment dropping all the climate data performs terribly.
Dropping the temperatures decreases accuracy significantly. Similarly,
though not as pronounced, dropping the fall data decreases accuracy.
When dropping the data for the summer on the other hand the error decreased.

\section{Discussion}

During the project it became apparent that the machine learning model gives features that differ between the
categories a higher importance than features that are largely identical.
Here we note that our interpretation was hampered by the strong correlation between features.
This correlation made itself noticeable in the low decreases in accuracy in the permutation importance plots
\ref{pl:histogramm_feature_imps3_basic} and \ref{pl:histogramm_feature_imps4_basic}.
Nonetheless figure \ref{pl:histogramm_feature_imps3_basic} showed that the soil cluster was
relatively unimportant.
On the other hand \ref{pl:egypt_shrub_desert_soil} showed that the soil features in the training set did not differ much.
The unimportance of soil features was a common theme during the project.
Similarly the cluster of the summer radiation was relatively unimportant.
Previously we remarked in the description of the climate diagrams
\ref{pl:climate:egypt:aridShrub} and \ref{pl:climate:egypt:desert}
that the radiation was quite similar in both biomes.

From the figure \ref{pl:s3_simulationComparisons_balancedErrorRate} produced by our experiment series 
we can get an idea of some measures that differ between the Egyptian and Libyan biomes. The accuracy 
increases when dropping spring, temperature or radiation. This means that at least for these measures there seems to have been some overfitting.
We note that according to figure \ref{pl:histogramm_feature_imps3_basic} they also have a big impact on the response of the model on the test set.

Regarding the performance of the multiclass classification we note that the large difference in the weighted accuracy and the accuracy
originates in the large variance of sample sizes for the respective biomes.
Additionally the small biomes tend to be classified poorly.
Because the classification problem at hand should treat every biome
equally the author believes that the balanced accuracy is more appropriate here.
We also note that the similarity between the countries almost certainly significantly improved the
performance of the model.

Since our model works well for \pyth{Biome_obs} and \pyth{Biome_Cmax} but performs terribly
if we test a model trained on \pyth{Biome_obs} with  \pyth{Biome_Cmax} we can conclude that the
\pyth{Biome_Cmax} classification is a bad approximation of \pyth{Biome_obs}. That the  \pyth{Biome_Cmax}
data is quite different from the  \pyth{Biome_obs} data is however quite apparent from a glimpse at the
maps \ref{pl:s4_basic_error_map} and \ref{pl:Basic_with_Biome_Cmax_error_map}.

In the regression part we see from the feature importance plot \ref{pl:histogramm_feature_imps5_npp_basic} that our model chose the temperature to be the most important feature to determine the
\pyth{NPP}. The plot \ref{pl:s5_simulationComparisons_MSE} showing the result for the experiment series on the other hand showed that the temperature is also a very good predictor.
Together this indicates that the model made a good choice when choosing the temperature to be the most important feature.
This also intuitively makes sense since temperature is a major factor when it comes to plant growth in the arctic.
% discuss shortcomings of models

\section{Conclusion}

We saw that the multiclass classification only performed well as long as the training and
test regions were
quite similar. This is a disadvantage of the random forest classifier since it can only
build on preexisting data and does not generalise well to new regions.
In this regard the random forest classifier acts a lot like an interpolator
and should not be used for extrapolation.
We also saw that the performance of the model depended a lot more on the
quality and quantity of the data that it is fed than the hyperparameters.
An advantage over more extensive models like LPG GUESS is that it is quickly set up.
It has however the disadvantage that it then is quite cumbersome to understand this model
and to gain valuable insights for human minds.


\section*{Bibliography}
\nocite{*}
%Main source
%\printbibliography[heading=none, keyword={main}]
%\noindent Other sources
\printbibliography[heading=none, keyword={secondary}]

\clearpage
\section{Appendix}
\subsection{General overview}

\begin{figure}[h]
  \centering
  \includegraphics[width=0.45\textwidth]{climate_libya.pdf}
  \caption{Climate diagram for Libya.}
  \label{fi:climate_libya}
\end{figure}

\begin{figure}[h]
  \begin{minipage}{0.45\textwidth}
    \centering
    \includegraphics[width=\textwidth]{climate_canada.pdf}
    \caption{Climate diagram for Canada.}
    \label{fi:climate_canada}
  \end{minipage}
  \hfill
  \begin{minipage}{0.45\textwidth}
    \centering
    \includegraphics[width=\textwidth]{climate_Russia.pdf}
    \caption{Climate diagram for Russia.}
    \label{fi:climate_Russia}
  \end{minipage}
\end{figure}

\begin{figure}[h]
  \centering
  \includegraphics[width=\textwidth]{Russia_Canada_cmax_map.pdf}
  \caption{Map of biomes \pyth{Cmax} in Russia and Canada.}
  \label{fi:}
\end{figure}

\clearpage

\subsection{Binary classification}

\begin{figure}[h]
  \centering
  \includegraphics[width=0.8\textwidth]{dendogramm_feature_imps3_basic.pdf}
  \caption{Dendogram to the binary classification.}
  \label{pl:dendogramm_feature_imps3_basic}
\end{figure}

\clearpage
\subsection{Multiclass classification}
\begin{figure}[h]
  \centering
  \includegraphics[width=0.8\textwidth]{dendogramm_feature_imps4_basic.pdf}
  \caption{Dendogram to the multiclass classification.}
  \label{pl:dendogramm_feature_imps3_basic}
\end{figure}


\begin{table}[h]
  \centering
  \input{../Project3RandomForestML/table/s4_basic_confTable.tex}
  \caption{Confusion table for the multiclass classification with \pyth{Biome_obs}.}
  \label{tb:s4_basic_confTable}
\end{table}
\begin{table}[h]
  \centering
  \input{../Project3RandomForestML/table/s4_basic_classreport.tex}
  \caption{Classreport for the multiclass classification with \pyth{Biome_obs}.}
  \label{tb:s4_basic_classreport}
\end{table}

\begin{table}[h]
  \centering
  \input{../Project3RandomForestML/table/s4Basic_with_Biome_Cmax_confTable.tex}
  \caption{Confusion table for the multiclass classification with \pyth{Biome_Cmax}.}
  \label{tb:s4Basic_with_Biome_Cmax_confTable}
\end{table}
\begin{table}[h]
  \centering
  \input{../Project3RandomForestML/table/s4Basic_with_Biome_Cmax_classreport.tex}
  \caption{Classreport for the multiclass classification with \pyth{Biome_Cmax}.}
  \label{tb:s4Basic_with_Biome_Cmax_classreport}
\end{table}


\clearpage
\subsection{Regression}
\begin{figure}[h]
  \centering
  \includegraphics[width=0.8\textwidth]{dendogramm_feature_imps5_npp_basic.pdf}
  \caption{Dendogram to the regression with \pyth{NPP}.}
  \label{pl:dendogramm_feature_imps5_npp_basic}
\end{figure}

\begin{figure}[h]
  \centering
  \begin{minipage}{0.45\textwidth}
    \centering
    \includegraphics[width=\textwidth]{s5_vegc_basic_regressionPlot.pdf}
    \caption{Predicted versus true values for the parameter VegC.}
    \label{pl:s5:vegc:basic:regressionPlot}
  \end{minipage}
  \hfill 
  \begin{minipage}{0.45\textwidth}
    \centering
    \includegraphics[width=\textwidth]{s5_vegc_basic_residualDistr.pdf}
    \caption{Distribution of the residuals for the parameter VegC.}
    \label{pl:s5:vegc:basic:residualDistr}
  \end{minipage}
\end{figure}

\end{document}
