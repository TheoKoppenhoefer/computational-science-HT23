
% Indicate to rubber that there are external files
% rubber: shell_escape


\input{../Latex_Templates/Preamble_Report}

%%%%% TITLE PAGE

%\subject{, VT23}
\title{ Report for the Course Modelling in Computational Science, HT23 \\[1ex]
	  \large Project 3: Biome classification}
%\subtitle{}
\author{Theo Koppenhöfer \\[1ex] (with Anna and Carmen, Group 4)}
\date{Lund \\[1ex] \today}

\addbibresource{bibliography.bib}

\graphicspath{{../Plots/}}

\pgfplotsset{
	compat=newest,
    every axis/.append style={
        axis y line=left,
        axis x line=bottom,
        scale only axis,
        % line width=2pt,
%    	max space between ticks=25pt,
        width=0.7\textwidth,
        scaled ticks=true,
        axis line style={thick,-,>=latex, shorten >=-.4cm},
    		x tick label style={
		    /pgf/number format/precision=3
		    }
    },
    every axis plot/.append style={very thick},
    tick style={black, thick},    
}


%%%%% The content starts here %%%%%%%%%%%%%

\usepackage{pythonhighlight}

\begin{document}

\maketitle

\section{Introduction}
The following report is part of the second project of the course Modelling in Computational Science, BERN01, taken at Lund university.
In this project we will use machine learning to classify biomes based on climate and soil data. We will test the performance of our machine
learning model in binary classification and in distinguishing multiple biomes for different regions. We will also compare our model with LPG\_guess output
and modify our model to predict continuous variables of LPG\_guess.
For this we will discuss the choice of regions and biomes, the setup of our model, give some interesting results, discuss these and finally give a conclusion.
The code to the project was implemented in \pyth{python}.
The project report and code can be found online under~\cite{Repository}.

\section{Methods}

To test our first binary classification model we chose the biomes `arid shrub' and `desert'. For the choice of regions
we had to choose two countries which contained sufficient amount of both regions. A plot of regions with sufficient amounts
of both biomes can be seen in figure 
\begin{figure}
  \centering
  \begin{minipage}{0.45\textwidth}
    \centering
    \missingfigure[figwidth=\textwidth]{}
    \caption{Data points with `desert' and `arid shrub' in selected countries.}
    \label{pl:}
  \end{minipage}
\end{figure}
Our initial choice was Egypt and China.
It turned out however that when we took out soil data our model could not handle the classification well since 
the deserts in both countries have very different climates. Thus we decided for Egypt for the training and Libya for the testing.

For the classification of multiple biomes we initially chose Africa and China but this quickly turned out to be a poor choice as
both regions have very different climate data. Thus we switched to the regions to Russia for training and Canada for testing.

For the regression model chose Canada to train and Russia to test the model. The reason for this switch of roles lies in the performance
of the training.

If not otherwise stated we use as training parameters all the parameters of the file \pyth{data_index_2.csv} excluding
\begin{python}
  ['MaxBiomeLAI','Biome_obs','Biome_LAI','Biome_Cmax',
                       'Lon','Lat','Pan_2007','ISO3','UN','MaxBiomeCmax',
                       'NPP','VegC','SoilC','LitterC','SoilR']
\end{python}

\section{Results}

\subsection{Binary classification}



\section{Discussion}

% discuss shortcomings of models
\newpage
\section{Conclusion}


\section*{Bibliography}
\nocite{*}
%Main source
%\printbibliography[heading=none, keyword={main}]
%\noindent Other sources
\printbibliography[heading=none, keyword={secondary}]


\end{document}
